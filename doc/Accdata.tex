\documentclass [10pt,  twoside] {amsart}
\usepackage [italian] {babel}
\usepackage [T1] {fontenc}
\usepackage [utf8] {inputenc}

\usepackage{geometry} % see geometry.pdf on how to lay out the page. There's lots.
\geometry{a4paper} % or letter or a5paper or ... etc
% \geometry{landscape} % rotated page geometry

% See the ``Article customise'' template for come common customisations

\usepackage{tikz}

\title{Tecniche di riconoscimento e visualizzazione}
%\subtitle{}
\author{Manuel Iurissevich}
\date{} % delete this line to display the current date

%%% BEGIN DOCUMENT
\begin{document}

\maketitle
\tableofcontents


%%% ACQUISIZIONE

\section{Acquisizione}
Le sequenze di esempio sono state registrate
con l'accelerometro triassiale di un telefono cellulare {\smaller GT-S7501},
grazie all'applicazione \emph{Accelerometer Monitor}.

Consistono in 16 movimenti di sollevamento
di cui i primi 8 sono considerati corretti,
gli altri errati.
I sollevamenti sono effettuati a gomito vincolato,
quindi soltanto con l'avambraccio.
Durante i movimenti ``giusti'' la traiettoria \`e mantenuta
il pi\`u possibile dritta, mentre durante quelli ``sbagliati''
si devia dall'asse verticale.

Il cellulare \`e tenuto in mano in maniera
che lo schermo guardi nella stessa direzione del palmo.
Il polso \`e rigido.

\begin{tikzpicture}
\end{tikzpicture}

\subsection{Specifiche tecniche}
Forse a questo punto non sono granch\'e interessanti.

\subsection{Il punto di vista biomeccanico}
Fissata che sia la posizione della punta del gomito,
l'avambraccio tre gradi di libert\`a di rotazione.
Il polso pu`o ruotare in altri due modi,
per un totale di cinque gradi di libert\`a.

In questa dimostrazione si cerca di sopprimere
i movimenti del polso e la rotazione sull'asse dell'avambraccio.
Fatto questo, si considera corretta l'esecuzione del movimento
quando non c'\`e variazione dell'angolo longitudinale
ma solo di quello latitudinale.\footnote{\Huge{Terminologia!!}}


\subsection{Elaborazione preliminare}
I dati grezzi in uscita dal sw di acquisizione
devono subire un trattamento
prima di passare attraverso qualunque algoritmo di classificazione.

\paragraph{Smoothing}
Si fa innanzitutto passare una media mobile per togliere un po' di rumore.
Buoni risultati con un filtro lungo 13 campioni.
Il filtro \`e applicato ``a mano'' con un ciclo $for$
e non con l'apposito comando $signal:filter$
perch\'e contemporaneamente si esegue un sottocampionamento
di fattore 3.
Si potrebbe ovviamente fare prima il filtraggio con $filter$
e comunque sottocampionare dopo
e credo anzi sia consigliato.

\paragraph{Suddivisione}
La seconda fase consiste nella divisione della registrazione
(che contiene tutti e 16 gli esempi)
in esempi separati.
Questo \`e fatto sia utilizzando tecniche software
che cercando le divisioni a mano.
Si fa in modo che tutte le sequenze abbiano una lunghezza di 64 campioni
ma non si bada al fatto che inizino tutte allo stesso istante.
Si calcolano quindi alcune caratteristiche statistiche dei dati acquisiti.
Basandoci su \emph{un lavoro precedente},
riteniamo che valga la pena concentrarsi su media e varianza.

\paragraph{Visualizzazione}
Siamo passati quindi dal rappresentare i dati
in uno spazio a $64\times3=192$ dimesioni
ad uno di $2*\times3=6$ dimensioni.
Per la visualizzazione dei dati in questa fase
si pu`o ricorrere alle proiezioni ortogonali
rappresentando due coordinate per volta
(per un totale di $\sum_{n=1}^{6-1} n =15$ grafici in 2D),
oppure ricorrere a tecniche di scalaggio multidimensionale
con il comando $statistics:mdscsale$
al quale diamo in pasto le distanze (euclidee) trai punti in 6D.
Abbiamo seguito quest'ultima via.

%%% TECNICHE CLASSICHE
\section{Tecniche classiche}
K-mean clustering, apprendimento competitivo, neurone.
\subsection{K-mean clustering}
Tecnica di classificazione non supervisionata
(non bisogna etichettare i dati prima di far partire l'algoritmo)
che consiste in
\begin{enumerate}
  \item{gettare pi\`u o meno casualmente \emph{K centroidi} nello spazio}
  \item{etichettare ogni punto con l'ID del centroide pi\`u vicino}
  \item{spostare ogni centroide sulla media dei punti a esso assegnati}
  \item{ripetere dal punto 2 finch\'e i centroidi non sono quasi fermi}
\end{enumerate}

\paragraph{Con Matlab}
Il toolbox $statistics$ mette a disposizione il comando $kmeans$ che,
fornitigli i punti e il numero di gruppi desiderati,
restituisce le etichette per ogni punto e le coordinate dei centroidi.
Altro scalaggio per visualizzare i dati in 2D.

\paragraph{Crossvalidazione}
Il numero di centri $K$ \`e fissato a priori
ma non sempre si hanno buoni motivi per scegliere un valore pittosto che un altro.
Per trovare il $K$ migliore si fa di solito correre l'algoritmo pi\`u volte,
ogni volta cambiando il $K$ e valutando le prestazioni.
L'insieme dei dati dev'essere preventivamente diviso
in un gruppo di addestramento e un gruppo di test, o validazione.


\subsection{Apprendimento competitivo}
Secondo me concettualmente simile al \linebreak\mbox{K-means},
\`e l'apprendimento competitivo.
(\emph{Attenzione!} Non sono la stessa cosa e non portano agli stessi risultati)

Consiste nello strutturare una rete di $K$ neuroni
ciascuno con il suo vettore dei pesi
inizialmente casuale ma sempre di modulo 1.
In questo modo a ogni neurone \`e assegnato un versore
in un qualche spazio.
I neuroni sono tra loro collegati tramite linee di segnali inibitori.
Quando le coordinate di un punto di addestramento
sono date in pasto alla rete,
ci sar\`a un neurone che indica in quella direzione,
o per lo meno un neurone sar\`a pi\`u vicino degli altri:
questo \`e l'unico neurone autorizzato ad apprendere
e modificher\`a i suoi pesi in maniera da avvicinarsi
un po' alla direzione del punto.
L'apprendimento degli altri neuroni \`e inibito.

Questa tecnica richiede ulteriori pretrattamenti dei dati in ingresso
in maniera che siano distribuiti (meglio possibile)
su una sfera centrata nell'origine.


\subsection{Il percettrone classico}
Il percettrone cerca sempre di dividere l'universo
mettendoci in mezzo un piano.
Tale piano \`e individuato dai pesi del neurone
e da una soglia, che in pratica \`e un ulteriore sinapsi
con l'ingresso sempre a 1 (o -1, se si preferisce).
La funzione di attivazione pu`o essere tranquillamente la funzione segno,
senza scomodare esponenziali o tangenti iperboliche.
Tanto c'\`e un solo neurone e non dobbiamo fare backpropagation.

Fatto sta che per ogni vettore d'ingresso
il percettrone calcola 
\begin{equation}
	y = sgn \left(x \cdot w - \theta \right);
	\label{eq:perceptron}
\end{equation}
in fase di addestramento, si confronta l'uscita calcolata
con quella desiderata: $e=d-y$
e si correggono i pesi di un fattore $\Delta w = a\,e\,x$,
dove $a \in [0, 1]$ \`e il fattore di apprendimento.
Analogo per la soglia.

Ad apprendimento concluso i pesi non variano pi\`u
ed il piano di separazione \`e finalmente individuato
dall'equazione \ref{eq:perceptron}, posta uguale a 0.





\subsection{La selezione delle caratteristiche}
Nonostante l'applicazione richieda di classificare,
uno degli aspetti pi\`u interessanti di tutto il discorso
\`e valutare le caratteristiche importanti che permettano
di classificare le sequenze con la massima efficacia.
Ridurre insomma al minimo le variabili e concentrarsi
su quelle che varamente portano informazione.

\begin{itemize}
	\item{quanto \`e utile questa caratteristica a farmi classificare i dati?}
	\item{quanto sono mescolati i dati lungo questa caratteristica (vista come coordinata)}
	\item{che \`e il contrario di ``quanto sono ordinati?''}
	\item{quanto mi \`e utile sapere che (su questa caratteristica) un dato viene prima di un altro?}
\end{itemize}

\paragraph{In una prima fase} si ordinano i valori secondo la caratteristica
e moltiplica le etichette ordinate per le etichette non riordinate.
La somma risultante (prod.scalare) d\`a l'importanza della caratteristica.

\paragraph{Il neuroncino} Ha un solo peso $w=1$
e deve soltanto variare la soglia.

\paragraph{A questo punto} Si possono dedicare attenzioni
soltanto alle caratteristiche pi\`u importanti,
oppure ``scartarle'' e concentrarsi su relazioni pi\`u sottili.
Bisogna fate poi il discorso delle caratteristiche dipendenti:
se due caratteristiche sono dipendenti,
non occorre considerarle entrambe.

\section{Tecniche moderne}
Dopo esserci concentrati sulle tecniche vintage,
consideriamo quelle in voga negli ultimi vent'anni.
Si attribuisce a un certo Vapnik l'applicazione fruttuosa
del kernel all'apprendimento automatico (1992).

\subsection{Il Kernel}
Prende il nome da qualche teoria matematica avanzata.
Per capirsi
\begin{itemize}
	\item{il prodotto scalare \`e un kernel e}
	\item{il kernel \`e un prodotto scalare.}
\end{itemize}

Dalla prima affermazione si pu`o intuire che,
presi due vettori, due elementi di uno spazio
in cui sia definito un prodotto scalare.
il kernel sar\`a un'estensione del prodotto scalare consueto.

La seconda introduce l'idea
di portarsi in un comodo spazio dotato di prodotto scalare
per poi applicare il kernel.

Formalmente:
\begin {equation}
  \label {eq:kernel}
  k = \langle \Phi(x_1), \Phi(x_2) \rangle
\end{equation}
dove $\Phi(x)$ \`e una mappa -se si vuole- non lineare
che porta i generici elementi $x$
in uno spazio dotato di prodotto scalare.

NB$_{1}$: Gli elementi originali possono quindi non essere neanche vettori.\par
NB$_{2}$: In generale � pi� semplice scrivere la funzione kernel piuttosto che la mappa.\par
Oss: Praticamente si stanno estraendo le caratteristiche.\par

\subsection{Il kernel e l'apprendimento automatico}
Rimanendo nell'ambito della classificazione binaria
la funzione discriminante \`e
\begin{equation}
	\label{eq:discriminante}
	y_n = sgn \left(\langle w, \Phi(x_n) \rangle + b \right)
\end{equation}
ma si definiscono i pesi come
\begin{equation}
  \label{eq:pesikernel}
  w = \sum_i y_i \alpha_i \Phi(x_i)
\end{equation}
con $\alpha_i$ che come si chiama?
Sostituendo la \ref{eq:pesikernel} nella \ref{eq:discriminante}
si ottiene
\begin{align}
  y_n &= sgn \left(\sum_i y_i \alpha_i \langle \phi_i,\phi_n \rangle + b \right) \\
  &= sgn \left(\sum_i y_i \alpha_i k(x_i,x_n) + b \right)
\end{align}

\subsection {Il kernel perceptron}
(Tratto da un articolo dell'Universit\`a di Trento).
L'algoritmo di apprendimento \`e molto simile
a quello del percettrone classico.
La differenza sta nel calcolo della funzione (vedi eq. precedente)
e nell'aggiornamento dei pesi:
gli $\alpha_i$ sono incrementati ogni volta che
il neurone sbaglia la previsione su un $x_i$.

\subsection {Simple Pattern Recognition Algorithm ({\smaller SPRA})}
Schoelkopf e Smola propongono un semplice algoritmo di classificazione
introduttivo alle Macchine a Vettori di Supporto ({\smaller SVM}).

Non si fa altro che prendere i punti medi delle due classi ($c_1$ e $c_2$),
prendere il vettore differenza trai due $w=c_1-c_2$
il loro punto medio ($c=(c1+c2)/2$).

Ora ogni altro punto di classe ignota
pu\`o essere etichettato con
$y =sgn \left( \langle \boldsymbol{x-c}, \boldsymbol{w} \rangle \right)$.

\paragraph{Osservazioni}


\subsection {Macchine a vettori di supporto ({\smaller SVM})}
Sono algoritmi che usano l'espediente del kernel
per mappare le sequenze in degli spazi comodi
(tipicamente con molte dimensioni pi\`u)
dove trovare il piano di separazione che massimizza il margine.
Nello spazio di partenza la separazione pu\`o assumere le forme pi\`u diverse
(dipende dalla mappa non lineare).

Il problema \`e posto nei termini di minimizzazione della lagrangiana
\begin{equation}
  \label{lagrangiana}
  L_P = lagrangiana
\end{equation}
o massimizzazione della sua duale
\begin{align}
  \label{lagrangiana}
  L_D &= \sum_i \alpha_i - \frac{1}{2} \sum_i \sum_j \alpha_i \,\alpha_j \,y_i \,y_j \,k(\boldsymbol{x_i}, \boldsymbol{x_j}) \\
  \langle \boldsymbol{\alpha}, \boldsymbol{y} \rangle &= 0 \\
  \alpha_i &\geq 0
\end{align}

\subsection{Prova di ottimizzazione}
Il progetto {\emph{ottimizzazione}} \`e una dimostrazione della funzione
\(fimncon\) di ottimizzazione vincolata.

\par
Questa strumento prende in ingresso
la funzione obiettivo e la funzione supporto,
vale a dire la funzione da minimizzare e i vincoli non lineari da rispettare;
il punto di partenza;
gli eventuali vincoli lineari di uguaglianza e disuguaglianza.


\subsection {I consigli di Chang e Lin}
I creatori della libreria $libsvm$ danno alcuni suggerimenti
per iniziare presto ad avere risultati soddifacienti con le {\smaller SVM}.
\begin{itemize}
  \item {trasformare i dati nel formato di un pacchetto svm}
  \item {eseguire uno scalaggio semplice dei dati}
  \item {considerare i kernel a base radiale (gaussiani)}
  \item {crossvalidare per trovare i $C$ e $\gamma$ migliori}
  \item {usare questi $C$ e $\gamma$ per addestrare sull'intero insieme di adestramento}
  \item {testare}
\end{itemize}

\paragraph {$C$ e $\gamma$}
Il parametro $C$ \`e il costo.
Serve a rendere pi\`u laschi i vincoli dell'ottimizzazione
\begin{equation}
  0 \leq \alpha_i \leq C
  \label{vincoliC}
\end{equation}
Necessario quando le due classi {\bfseries non} sono separabili,
\`e utile quando le classi sono separabili in maniera "forzata"
cio\`e l'insieme di addestramento le mostra separabili
ma dagli insiemi di validazione o test appare che non lo sono.
Il parametro $\gamma$ fa riferimento alla formula del kernel a base radiale ({\smaller RBFK})
\begin{equation}
  k(\boldsymbol{x_1}, \boldsymbol{x_1}) = 
    \gamma\,||\boldsymbol{x_1}-\boldsymbol{x_2}||^2
  \label{rbfk}
\end{equation}
ma non \`e utilizzato nell'esempio
perch\'e invece del kernel a base radiale
se ne usa uno polinomiale di grado $2$.

\subsection {Esempio di {\smaller SVM}}
Nell'esempio $svmxor$
\footnote{\, Il nome deriva dal fatto
che il primo problema posto alla macchina
era uno {\smaller XOR}}
si fanno quindi due passate con una griglia di 5 elementi,
prima con $C\in[10^{-5}, \dots, 10^{5}]$
poi intorno al $C$ che si \`e mostrato migliore.

Si chiama quindi la funzione di addestramento $my\_svmtrain$
che per ogni $C$ chiama a sua volta $fmincon$.
Terminato l'addestrameto si calcola sull'insieme di validazione
la percentuale di previsioni esatte,
indicativa della bont\`a del parametro $C$ usato in quel ciclo.

Si fa girare un'altra volta l'apprendimento con il $C$ migliore trovato
e si collauda il tutto sull'insieme di test.
Il resto del documento \`e dedicato alla visualizzazione
dei dati e del processo di apprendimento.

Si \`e usato l'insieme di dati "Iris".






\end{document}



